%%%%%%%%%%%%%%%%%%%%%%%%%%%%%%%%%%%%%%%%%%%%%%%%%%%%%%%%%%%%%%%%%%
%%%%%%%%%%%%%%%%%%%%%%%%%%%%%%%%%%%%%%%%%%%%%%%%%%%%%%%%%%%%%%%%%%
\documentclass[12pt,letterpaper]{article}
%\documentclass[12pt]{amsart}
% use package for extended size
%\documentclass[landscape,20pt]{extarticle}
%%%%%%%%%%%%%%%%%%%%%%%%%%%%%%%%%%%%%%%%%%%%%%%%%%%%%%%%%%%%%%%%%%
%%%%%%%%%%%%%%%%%%%%%%%%%%%%%%%%%%%%%%%%%%%%%%%%%%%%%%%%%%%%%%%%%%

%%%%%%%%%%%%%%%%%%%%%%%%%%%%%%%%%%%%%%%%%%%%%%%%%%%%%%%%%%%%%%%%%%
%%%COMMANDS %%%%COMMANDS %%%%COMMANDS %%%%COMMANDS %%%%COMMANDS%%%
%%%%%%%%%%%%%%%%%%%%%%%%%%%%%%%%%%%%%%%%%%%%%%%%%%%%%%%%%%%%%%%%%%

%%%%%%%%%%%%%%%%%%%%%%%%%%%
% include graphics package
%%%%%%%%%%%%%%%%%%%%%%%%%%%
\usepackage{graphics,epsfig,graphicx,color}
\graphicspath{{/EPSF/}{../Figures/}{Figures/}}
\usepackage[caption = false]{subfig}
\usepackage{float}
\usepackage{capt-of}
%\usepackage{subfigure}


%%%%%%%%%%%%%%%%%%%%%%%%%%%
% include various packages
%%%%%%%%%%%%%%%%%%%%%%%%%%%

% include symbols package
\usepackage{amssymb,latexsym}

% include setspace package
\usepackage{setspace}

% include ams math package
\usepackage{amsmath}

% include fonts package
\usepackage{mathrsfs,amsfonts}

\usepackage{mathtools}

% include ams theorem package
\usepackage{amsthm,amsxtra}

% include the showkeys package
%\usepackage{showkeys}
\usepackage[final]{showkeys}

% include the stmaryrd package
% that contains many math symbols
\usepackage{stmaryrd}

\usepackage{fullpage}

\title{Introduction To The \\Local Discontinuous Galerkin Method}
\author{Michael Harmon}


\begin{document}


\maketitle

We discuss the local discontinuous Galerkin (LDG) method to approximate solutions of the Poisson equation:
\begin{equation}
\begin{aligned}
- \nabla \cdot  \left(\ \nabla u \ \right)&= f(\textbf{x}) && \mbox{in} \ \Omega, \\
-\nabla u \cdot \textbf{n}  &= g_{N}(\textbf{x}) && \mbox{on} \ \partial \Omega_{N} \\
u &= g_{D}(\textbf{x}) && \mbox{on}  \ \partial \Omega_{D}.
\end{aligned}
\end{equation}


We first introduce some notation. Let $\mathcal{T}_{h} = \mathcal{T}_{h}(\Omega) \, = \, \left\{ \, \Omega_{e} \, \right\}_{e=1}^{N}$ be  the general triangulation of a domain $\Omega \; \subset \; \mathbb{R}^{d}, \; d \, = \, 1, 2, 3$, into $N$ non-overlapping elements $\Omega_{e}$ of diameter $h_{e}$.  The maximum size of the diameters of all the elements is $h = \max( \, h_{e}\, )$.  We define $\mathcal{E}_{h}$ to be the set of all element faces and $\mathcal{E}_{h}^{i} $ to be the set of all interior faces of elements which do not intersect the total boundary $(\partial \Omega)$. We define $\mathcal{E}_{D}$ and $\mathcal{E}_{N}$ to be the sets of all element faces and on the Dirichlet and Neumann boundaries respectively. Let $\partial \Omega_{e} \in \mathcal{E}_{h}^{i}$ be a interior boundary face element, we define the unit normal vector to be,
\begin{equation}
\textbf{n} \; = \; \text{unit normal vector to } \partial \Omega_{e}  \text{ pointing from } \Omega_{e}^{-} \, \rightarrow  \, \Omega_{e}^{+}.
\end{equation}


\noindent
We take the following definition on limits of functions on element faces,
\begin{align}
w^{-} (\textbf{x} ) \, \vert_{\partial \Omega_{e} } \; = \; \lim_{s \rightarrow 0^{-}} \, w(\textbf{x}  +  s  \textbf{n}),  && w^{+} (\textbf{x} ) \, \vert_{\partial \Omega_{e} } \; = \; \lim_{s \rightarrow 0^{+}} \, w(\textbf{x}  + s  \textbf{n}).
\end{align} 
\noindent
We define the average and jump of a function across an element face as,
\begin{equation}
\{f\} \; = \; \frac{1}{2}(f^-+f^+), 
\qquad \mbox{and} \qquad 
\llbracket f \rrbracket \; = \; f^+ \textbf{n}^+ + f^- \textbf{n}^-,
\end{equation}
\noindent
and,
\begin{equation}
\{\textbf{f} \} \; = \; \frac{1}{2}(\textbf{f}^- + \textbf{f}^+), 
\qquad \mbox{and}\qquad  
\llbracket \textbf{f} \rrbracket \; = \;\textbf{f}^+ \cdot \textbf{n}^+ + \textbf{f}^- \cdot \textbf{n}^- , 
\end{equation}
\noindent
where $f$ is a scalar function and $\textbf{f}$ is vector-valued function. We note that for a faces that are on the boundary of the domain we have,
\begin{equation}
\llbracket f \rrbracket \; = \; f \,  \textbf{n} 
\qquad \mbox{and}\qquad  
\llbracket \textbf{f} \rrbracket \; = \; \textbf{f} \cdot \textbf{n}.
\end{equation}

\noindent
We denote the volume integrals and surface integrals using the $L^{2}(\Omega)$ inner products by $( \, \cdot \, , \, \cdot \, )_{\Omega}$ and $\langle  \, \cdot \, , \, \cdot \,  \rangle_{\partial \Omega}$ respectively. 
\vspace{2mm}

As with the mixed finite element method, the LDG discretization requires the Poisson equations be written as a first-order system.  We do this by introducing an auxiliary variable which we call the current flux variable $\textbf{q}$:
\begin{align}
\label{eq:Primary}
\nabla \cdot \textbf{q}_{n}
\; &= \; 
f(\textbf{x}) && \text{in} \ \Omega,  \\
\label{eq:Auxillary}
\textbf{q}_{n} 
\; &= \;
 -\nabla u && \text{in} \ \Omega,   \\
\textbf{q}_{n}   \cdot \textbf{n} 
\; &= \; g_{N}(\textbf{x}) && \text{on} \ \partial \Omega_{N},\\
u &= g_{D}(\textbf{x}) && \mbox{on}\ \partial \Omega_{D}.
\end{align}

In our numerical methods we will use approximations to scalar valued functions that reside in the finite-dimensional broken Sobolev spaces,
\begin{align}
W_{h,k}
\, &= \, 
\left\{ w \in L^{2}(\Omega) \, : \; w  \vert_{\Omega_{e}} \in \mathcal{Q}_{k,k}(\Omega_{e}), \quad \forall \, \Omega_{e}  \in \mathcal{T}_{h}  \right\}, 
\end{align}

\noindent
where $\mathcal{Q}_{k,k}(\Omega_{e})$ denotes the tensor product of discontinuous polynomials of order $k$ on the element $\Omega_{e}$. We use approximations of vector valued functions that are as,
\begin{align}
\textbf{W}_{h,k} 
\, &= \, 
\left\{  \textbf{w}  \in \left(L^{2}(\Omega)\right)^{d} \, :
 \; \textbf{w}  \vert_{\Omega_{e}} \in \left( \mathcal{Q}_{k,k}(\Omega_{e}) \right)^{d}, \quad \forall \, \Omega_{e} \in  \mathcal{T}_{h} \right\}
\end{align}

\noindent
We seek approximations for densities $u_{h} \in W_{h,k}$ and gradients $\textbf{q}_{h}\in \textbf{W}_{h,k}$. Multiplying \eqref{eq:Primary} by $w \in W_{h,k}$ and \eqref{eq:Auxillary} by $\textbf{w} \in \textbf{W}_{h,k}$ and integrating the divergence terms by parts over an element $\Omega_{e} \in \mathcal{T}_{h}$ we obtain,
\begin{align}
-
\left( \nabla w  \, , \, \textbf{q}_{h}  \right)_{\Omega_{e}}
+
\langle w \, , \,  \textbf{q}_{h}  \rangle_{\partial \Omega_{e}} 
\ &= \
\left( w , \, f \right)_{\Omega_{e}} , \nonumber \\
\left( \textbf{w} \, , \, \textbf{q}_{h} \right)_{\Omega_{e}}
-
\left(  \nabla \cdot \textbf{w} \, , \,  u_{h} \right)_{\Omega_{e}}
+
\langle   \textbf{w}  \, ,   \, u_{h}
\rangle_{\partial \Omega_{e}} 
\ &= \
0 ,\nonumber
\end{align}

\noindent
Summing over all the elements leads to the \textbf{weak formulation}:

\vspace{2mm}

\noindent
Find $u_{h} \in W_{h,k}$ and $\textbf{q}_{h} \in  \textbf{W}_{h,k} $ such that,
\begin{align}
-
\sum_{e} \left( \nabla w,  \,  \textbf{q}_{h}  \right)_{\Omega_{e}}
 +
\langle  \llbracket \,  w \, \rrbracket \, , \,  \widehat{\textbf{q}_{h} } \rangle_{\mathcal{E}_{h}^{i} }
 +
\langle  \llbracket \,  w \, \rrbracket \, , \,  \widehat{\textbf{q}_{h} } \rangle_{\mathcal{E}_{D} \cup \mathcal{E}_{N}} \ &= \  
\left( w , \, f  \right)_{\Omega} \label{eq:LDG1}  \\
\sum_{e} \left( \textbf{w} \, , \, \textbf{q}_{h} \right)_{\Omega_{e}}
-
\sum_{e} \left(  \nabla \cdot \textbf{w} , \,  u_{h} \right)_{\Omega_{e}}
+  
\langle \, \llbracket \,  \textbf{w} \, \rrbracket \, ,   \, \widehat{u_{h}}
\rangle_{\mathcal{E}_{h}^{i}}
+ 
\langle  \llbracket \,  \textbf{w} \, \rrbracket \, , \,  \widehat{u_{h}}  \rangle_{\mathcal{E}_{D} \cup \mathcal{E}_{N}} \label{eq:LDG2}
\ &= \
0
\end{align}

\noindent
for all $(w,\textbf{w}) \in W_{h,k} \times \textbf{W}_{h,k}$.


\vspace{2mm}

\noindent
The terms $\widehat{\textbf{q}_{h}}$ and $\widehat{u_{h}}$ are the numerical fluxes. The numerical fluxes are introduced to ensure consistency, stability, and enforce the boundary conditions weakly.  The flux $\widehat{u_{h}}$ is,
\begin{equation} \label{eq:UFLUX}
\widehat{u_{h}} \; = \; \left\{
\begin{array}{cl}
\left\{ u_{h} \right\} \ + \ \boldsymbol \beta \cdot \llbracket u_{h} \rrbracket &  \ \text{in} \  \mathcal{E}_{h}^{i} \\
u_{h} &  \ \text{in} \ \mathcal{E}_{N}\\
g_{D}(\textbf{x})  & \ \text{in} \ \mathcal{E}_{D} \\
\end{array}
\right.
\end{equation}


\noindent
The flux $\widehat{\textbf{q}_{h}}$ is,
\begin{equation} \label{eq:QFLUX}
\widehat{\textbf{q}_{h}}  \; = \; \left\{
\begin{array}{cl}
\left\{ \textbf{q}_{h} \right\} \ - \  \llbracket \textbf{q}_{h} \, \rrbracket \boldsymbol \beta \ + \ \sigma \, \llbracket \, u_{h} \, 
\rrbracket & \ \text{in} \ \mathcal{E}_{h}^{i} \\
g_{N}(\textbf{x}) \, \textbf{n} \,  & \ \text{in} \ \mathcal{E}_{N}\\
\textbf{q}_{h} \ + \ \sigma \, \left(u_{h} - g_{D}(\textbf{x}) \right) \, \textbf{n} & \ \text{in} \ \mathcal{E}_{D} \\
\end{array}
\right.
\end{equation}

\noindent
The term $\boldsymbol \beta$ is a constant vector which does not lie parallel to any element face in $ \mathcal{E}_{h}^{i}$.  For $\boldsymbol \beta =  0$,  $\widehat{\textbf{q}_{h}}$ and $\widehat{u_{h}}$ are called the central or Brezzi et. al. fluxes. For  $\boldsymbol \beta \neq  0$, $\widehat{\textbf{q}_{h}}$ and $\widehat{u_{h}}$ are called the LDG/alternating fluxes.  The term $\sigma$ is the penalty parameter that is defined as,
\begin{equation}
\sigma \; = \; \left\{ 
\begin{array}{cc}
\tilde{\sigma} \, \min \left( h^{-1}_{e_{1}}, h^{-1}_{e_{2}} \right) & \textbf{x} \in \langle \Omega_{e_{1}}, \Omega_{e_{2}} \rangle \\
\tilde{\sigma}  \, h^{-1}_{e} & \textbf{x} \in \partial \Omega_{e} \cap \in \mathcal{E}_{D}
\end{array}
\right. 
\label{eq:Penalty}
\end{equation}

\noindent
with $\tilde{\sigma}$ being a positive constant.

\vspace{2mm}

We can now substitute \eqref{eq:UFLUX} and \eqref{eq:QFLUX} into \eqref{eq:LDG1} and \eqref{eq:LDG2} to obtain the solution pair $(u_{h}, \textbf{q}_{h})$ to the semi-discrete LDG approximation to the drift-diffusion equation given by:

\vspace{2mm}

 
Find $u_{h} \in W_{h,k}$ and $\textbf{q}_{h} \in  \textbf{W}_{h,k}$ such that,
\begin{align}
a(\textbf{w}, \textbf{q}_{h})  \ +  \ b^{T}(\textbf{w}, u_{h}) \ &= \ G(\textbf{w}) \nonumber \\
b(w, \textbf{q}_{h}) \ + \   c(w, u_{h})  \ &= \ F(w)
\label{eq:LDG_bilinear}
\end{align}
\noindent
for all $(w, \textbf{w}) \in W_{h,k} \times \textbf{W}_{h,k}$.  This leads to the linear system,

\begin{equation}
\left[ 
\begin{matrix}
A  & -B^{T}  \\
B & C 
\end{matrix}
\right]
\left[
\begin{matrix}
\textbf{U}\\
\textbf{Q}
\end{matrix}
\right]
\ = \
\left[
\begin{matrix}
\textbf{G}\\
\textbf{F}
\end{matrix}
\right]
\end{equation}
Where $\textbf{U}$ and $\textbf{Q}$ are the degrees of freedom vectors for $u_{h}$ and $\textbf{q}_{h}$ respectively. The terms $\textbf{G}$ and $\textbf{F}$ are the corresponding vectors to $G(\textbf{w})$ and $F(w)$ respectively. The matrix in for the LDG system is non-singular for any $\sigma > 0$.

\vspace{2mm}

The bilinear forms  in \eqref{eq:LDG_bilinear} and right hand functions are defined as,

\begin{align}
b(w, \textbf{q}_{h}) \, &= \,
-
\sum_{e}  \left(\nabla w, \textbf{q}_{h} \right)_{\Omega_{e}}
+
\langle \llbracket w \rrbracket, 
\left\{\textbf{q}_{h} \right\} - \llbracket \textbf{q}_{h} \rrbracket \boldsymbol \beta \rangle_{\mathcal{E}_{h}^{i}} 
 + 
\langle w, \textbf{n} \cdot \textbf{q}_{h} \rangle_{\mathcal{E}_{D}}
\\
%%%%%%%%%%%%%%%%%%%%%%%%%%%%%%%%%%%%%%%%%%%%%%%%%%%%%%%%%%%%%%%%%%%%%%%%%%%%%%%
a(\textbf{w},\textbf{q}_{h}) \, &= \,  
\sum_{e} \left(\textbf{w}, \textbf{q}_{h} \right)_{\Omega_{e}} \\
%%%%%%%%%%%%%%%%%%%%%%%%%%%%%%%%%%%%%%%%%%%%%%%%%%%%%%%%%%%%%%%%%%%%%%%%%%%%%%%
-b^{T}(w, \textbf{q}_{h}) \, &= \,
-
\sum_{e}  \left(\nabla \cdot \textbf{w}, u_{h} \right)_{\Omega_{e}} 
+ 
\langle  \llbracket \textbf{w} \rrbracket, \left\{u_{h} \right\} + \boldsymbol \beta \cdot \llbracket u_{h} \rrbracket \rangle_{\mathcal{E}_h^{i} }
+
\langle w, u_{h} \rangle_{\mathcal{E}_{N} } \\
%%%%%%%%%%%%%%%%%%%%%%%%%%%%%%%%%%%%%%%%%%%%%%%%%%%%%%%%%%%%%%%%%%%%%%%%%%%%%%%
c(w,u_{h}) \, &= \,
\langle \llbracket w \rrbracket,  \sigma \llbracket u_{h} \rrbracket \rangle_{\mathcal{E}_{h}^{i}}
+
\langle  w, \sigma u_{h} \rangle_{\mathcal{E}_{D}}  \\
%%%%%%%%%%%%%%%%%%%%%%%%%%%%%%%%%%%%%%%%%%%%%%%%%%%%%%%%%%%%%%%%%%%%%%%%%%%%%%%
G(\textbf{w}) \ & = \ - \langle \textbf{w}, g_{D} \rangle_{\mathcal{E}_{D}}\\
%%%%%%%%%%%%%%%%%%%%%%%%%%%%%%%%%%%%%%%%%%%%%%%%%%%%%%%%%%%%%%%%%%%%%%%%%%%%%%%
F(w) \ & = \  (w,f) - \langle w, g_{N} \rangle_{\mathcal{E}_{N}}  + \langle w, \sigma g_{D} \rangle_{\mathcal{E}_{D}} 
%%%%%%%%%%%%%%%%%%%%%%%%%%%%%%%%%%%%%%%%%%%%%%%%%%%%%%%%%%%%%%%%%%%%%%%%%%%%%%%
\end{align}

For more information, see the deal.ii code-gallery website.
\end{document}